\newpage
\section{Renato Lerac Corrêa de Sá}
\label{sec:renato_subjetiva}

O mundo da informática sempre fez parte da minha vida. Na época de garoto, jogos eletrônicos eram o
meu passatempo predileto e, ao ganhar meu primeiro computador, minha curiosidade sobre como tudo
funcionava despertou. Pesquisando por conta própria, aprendi sobre os diversos componentes que
constituem uma máquina moderna e a relação entre eles. A partir deste ponto, comecei a montar meus
próprios computadores, peça por peça, sempre buscando as últimas inovações tecnológicas.

Devido ao meu interesse por hardware, ingressei na Escola Politécnica da USP com o objetivo de
cursar Engenharia da Computação. Foi então que tive o primeiro contato com programação e desde então
meu interesse por softwares apenas aumentou.

Aliando meu novo interesse ao grande crescimento pelo qual a área de tecnologia da informação
vem passando nos últimos anos e a falta de profissionais qualificados no mercado, optei por
abandonar o curso de engenharia e ingressar no curso de Bacharelado em Ciências da Computação do
IME.

\subsection{IME/BCC}

Estou prestes a me formar e agora consigo enxergar o curso de uma forma mais abrangente. Sinto que o
instituto forneceu-me uma excelente base teórica, de modo que hoje possuo conhecimento sobre as mais
diversas áreas relacionadas à tecnologia da informação. Acredito que este deveria ser o objetivo de
qualquer curso de graduação, isto é, proporcionar ao graduando um aprendizado amplo sobre sua área,
de forma que ele tenha conhecimento suficiente para escolher uma especialização ao fim do curso, se
desejar.

\subsection{Disciplinas}

A grande maioria das disciplinas cursadas foi essencial à minha formação e, direta ou
indiretamente, importante no desenvolvimento deste trabalho. Faço, no entanto, uma lista
com as disciplinas que considero mais importantes. Primeiramente, gostaria de citar as disciplinas
que fundamentam a base do curso:

\begin{itemize}
	\item Cálculo Diferencial e Integral e Álgebra: quase todas as áreas do curso requerem algum
	conhecimento matemático. Estas disciplinas dão a base necessária para que o graduando
	sinta-se confortável ao lidar com matemática;
	\item Estatística e Processos Estocásticos: probabilidades e distribuições estão presentes
	em diversas disciplinas e conhecimento em estatística é fundamental para qualquer
profissional qualificado;
	\item Introdução à Computação: é a disciplina que ensina o graduando a raciocinar através
	de algoritmos, sendo, portanto, a porta de entrada do curso.
\end{itemize}

Algumas disciplinas foram importantes devido ao conhecimento conceitual proporcionado. Entre elas,
destaco:

\begin{itemize}
	\item Princípios de Desenvolvimento de Algoritmos: introduz diversos conceitos necessários
	para a programação, entre eles: recursividade, lista ligada, busca binária e ordenação.
	Promove uma primeira abordagem sobre a eficiência de algoritmos;
	\item Estrutura de Dados: discute métodos eficientes para abstrair entidades e dados reais,
	o que é parte fundamental de qualquer projeto de computação;
	\item Análise de Algoritmos: através de uma abordagem mais intensa sobre a eficiência dos
	algoritmos, introduz conceitos importantes como recorrências, programação dinâmica e
	algoritmos gulosos;
	\item Algoritmos em Grafos: é uma das disciplinas mais interessantes e úteis do curso, já
	que grafos estão presentes em diversos problemas reais;
	\item Programação Linear: apesar de ser o pesadelo de muitos alunos, o conhecimento
	adquirido na disciplina foi importante durante a realização deste trabalho de formatura;
	\item Programação Concorrente: devido à tecnologia multi-core existente hoje, torna-se
	necessário aprender conceitos de concorrência.
\end{itemize}

Um aspecto importante do curso é a prática de programação. Muitas disciplinas exigem
exercícios-programas, mas algumas se destacam por oferecerem experiências únicas ao graduando:

\begin{itemize}
	\item Laboratório de Programação I e II: porporcionam ao aluno as primeiras oportunidades
	para trabalhar em equipe e dedicar-se a projetos mais complexos do que os
	exercícios-programas habituais;
	\item Engenharia de Software: permite ao aluno aplicar conceitos de métodos ágeis em um
	projeto de programação web;
	\item Laboratório de Programação Extrema: é a melhor disciplina do curso. O aluno trabalha
	em projetos reais com sua equipe, respondendo diretamente a um cliente. Tudo feito em um
	ambiente descontraído;
	\item Trabalho de Formatura Supervisionado: foi muito bom ter como última experiência no
	curso a chance de unir todo o conhecimento adquirido em 4 anos de luta em um extenso projeto
	de pesquisa e desenvolvimento. Além disso, foi interessante conhecer os trabalhos dos
	colegas	de curso.
\end{itemize}

Por fim, devido ao meu grande interesse por hardware, gostaria de mencionar duas disciplinas que,
embora não sejam fundamentais para a formação, foram de grande valor:

\begin{itemize}
	\item Sistema Operacionais: por desvendar a mágica por trás dos sistemas operacionais
	modernos e esclarecer como é realizada a comunicação entre software e hardware;
	\item Organização de Computadores: por mergulhar no mundo do hardware e apresentar 
	diferentes arquiteturas.
\end{itemize} 

\subsection{Projeto do TCC}

Durante estes 4 anos de curso, desenvolvi uma amizade muito grande com os co-autores deste trabalho:
Daniel e Lucas. Em todas as disciplinas que cursamos juntos, sempre realizamos trabalhos e estudos
em equipe. Foi natural que decidíssimos desenvolver o TCC juntos, apesar da maior cobrança a que um
trabalho em equipe está sujeito.

Como o Daniel é comandante em uma companhia aérea, decidimos que seria interessante fazer um
trabalho relacionado. Nosso orientador Alfredo pediu desde o início que realizássemos um projeto
considerando a realidade brasileira. O Daniel então sugeriu o problema da otimização de viagens e
nosso orientador ficou logo entusiasmado.

Foram necessárias várias semanas de pesquisa para que obtivéssemos o conhecimento necessário sobre o
problema. Todavia, quanto mais pesquisávamos o assunto, maior se tornava o interesse da equipe.
Então, junto com nosso orientador Alfredo, optamos por bater o martelo, na certeza de que o projeto
seria muito trabalhoso, mas que no fim valeria a pena.

Gostaria de agredecer o Professor Alfredo, cuja ajuda durante estes longos meses de pesquisa e
desenvolvimento foi importantíssima. Acho que cumprimos nossos objetivos e espero que possamos
escrever o artigo sugerido pelo Professor.

\subsection{Comentários Finais}

Apesar da qualidade da formação, acho que é importante ressaltar alguns pontos negativos. Em
primeiro lugar, a cobrança é excessiva em alguns casos. A maioria das disciplinas demanda uma
enorme quantidade de horas dedicadas fora dos horários de aula. Isto é um problema ainda maior para
quem trabalha ou por algum outro motivo não possui tanto tempo livre para se dedicar aos estudos.

Em segundo lugar, é complicado cobrar o conteúdo de algumas disciplinas em provas escritas. A
necessidade de provas em tais disciplinas poderia ser revista. Como exemplo, posso citar Engenharia
de Software e Laboratório de Programação.

Por último, acredito que o trabalho em equipe pode ser mais estimulado ao longo do curso. É
importante realizar programação pareada nos EPs, mas são poucas as disciplinas em que
realmente podemos trabalhar em equipe.

Enfim, vale dizer que foi necessária muita dedicação para chegar ao final do curso. Os professores
são excelentes profissionais e sempre estiveram dispostos a ajudar. Saio do IME com uma base de
conhecimento sólida e abrangente, que me permitirá realizar trabalhos em diferentes áreas.


