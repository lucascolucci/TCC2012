\chapter*{Abstract}
\thispagestyle{empty}

A pairing is defined as a linked sequence of flights, satisfying a series of legislation imposed
restrictions, which initiates and ends at the crew contractual base. In the pairings determination
problem (PDP), the input is a provided flight list and the output should be a partition of that list
with a legal pairing set. The partition is done so it minimizes the cost, with the condition that
each flight is just once covered by some pairing of the solution. The PDP can be formulated as a
NP-hard combinatorial optimization problem, known as set partition. In this monograph we analyze the
PDP in the brazilian context, applying the brazilian rules and cost structure to the flight
generation. We solve this problem for some test data provided by brazilian airlines. We implemented
and applied three studied heuristics: a local search procedure, a hybrid genetic algorithm and a
column generation method. The obtained results analysis indicates good quality of the encountered
solutions.


