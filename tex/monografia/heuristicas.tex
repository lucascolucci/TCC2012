\zerar
\chapter{Heurísticas}
\label{cap:heuristicas}


%%%%%%%%%%%%%%%%%%%%%%%%%%%%%%%%%%%%%%%%%%%%%%%%%%%%%%%%%%%%%%%%%%%%%%%%%%%%%%%%%%%%%%%%%%%%%%%%%%%%

\section{Busca Local}
\label{sec:metodos_busca}

O método de busca local é bastante simples e foi um dos primeiros a ser utilizado na tentativa de 
melhorar uma solução viável do problema (\ref{eq:scpdv}). 

A busca local consiste em escolher aleatoriamente duas ou três viagens da solução viável inicial e, 
a partir da lista de etapas cobertas por essas viagens, gerar explicitamente todas as possíveis 
viagens legais usando o gerador. Como o número de etapas não é muito grande, o número de variáveis 
geradas é gerenciável. O modelo (\ref{eq:scpdv}) é então resolvido para todas essas variáveis, 
obtendo um novo conjunto de viagens que cobre o lista de etapas inicial. 

Se o custo desse novo conjunto de viagens for menor do que o original, então as viagens originais
serão substituídas na solução global. O processo é iterado um número máximo de vezes (ou um tempo 
máximo de execução), ou até que não haja variação significativa do custo (mínimo local), de tal 
forma que o custo sempre seja reduzido a cada passo.

%%%%%%%%%%%%%%%%%%%%%%%%%%%%%%%%%%%%%%%%%%%%%%%%%%%%%%%%%%%%%%%%%%%%%%%%%%%%%%%%%%%%%%%%%%%%%%%%%%%%

\section{Algoritmo Genético}
\label{sec:metodos_genetico}

Daremos aqui a formulação de um algoritmo genético para o problema (\ref{eq:scpdv}). 

%%%%%%%%%%%%%%%%%%%%%%%%%%%%%%%%%%%%%%%%%%%%%%%%%%%%%%%%%%%%%%%%%%%%%%%%%%%%%%%%%%%%%%%%%%%%%%%%%%%%

\section{Geração de Colunas}
\label{sec:metodos_colunas}

Nesta seção descreveremos o método de geração de colunas para obtenção de uma solução ótima 
associada ao problema em (\ref{eq:scpdv}), chamado de problema mestre.

A ideia básica consiste em não gerar explicitamente todas as $n$ viagens possíveis do problema. 
Ao invés disso, começamos com um número reduzido de colunas que forneça ao menos uma solução viável
(o chamado problema restrito). Essas variáveis são levadas ao otimizador considerando a versão 
relaxada de (\ref{eq:scpdv}). Para determinar se o problema original com todas as $n$ variáveis 
está resolvido otimamente, solucionamos o seguinte subproblema:
%
\begin{equation} \label{eq:sub}
	w^\ast = \min_{j = 1, \ldots, n} \left[ c_j - \sum_{i=1}^m \pi_i a_{ij} \right] \ev
\end{equation} 
%
onde $\pi_i$, $i = 1, \ldots, m$ são as variáveis duais ótimas associadas ao problema restrito.
Da teoria da programação linear, sabemos que se $w^\ast \geq 0$ então o problema restrito é ótimo
ao problema mestre. Por outro lado, se  $w^\ast < 0$, a coluna $k$ (com custo reduzido 
$\bar{c}_k < 0$) é identificada e é adicionada ao problema restrito. O problema restrito é 
resolvido novamente e o processo se repete até que nenhuma variável de custo reduzido negativo 
seja encontrada.

Ocorre que o subproblema (\ref{eq:sub}) pode ser traduzido como um problema de caminho mais curto 
no grafo da rede de voos, o qual pode ser resolvido de forma eficiente. Mostraremos isso com mais 
detalhes na próxima versão deste trabalho.

