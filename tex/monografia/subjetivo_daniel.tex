\section{Daniel Augusto Cortez}
\label{sec:daniel_subjetiva}

O meu interesse pela programação surgiu cedo quando aos 14 anos escrevi meu primeiro programa na
linguagem BASIC estudando um livro velho do meu pai da sua época de Poli. Meu interesse até então,
limitou-se a simples curiosidade. Aos 18 anos não quis me ``limitar'' à programar, então resolvi
estudar física. Aos 21 me formei bacharel pelo IFUSP, tendo escrito alguns poucos programas em C e
FORTRAN durante a graduação. Segui com a pós-gradução e fui trabalhar em outra área. A programação
foi deixada um pouco de lado durante esses anos.

Em meu trabalho me deparei com a dificuldade que as pessoas tinham em automatizar tarefas manuais
por falta de conhecimento. Elas ficavam espantadas quando eu apresentava um programa escrito por mim
que pudesse auxiliá-las. Fui escrevendo meus programinhas, mas sentia falta de uma base conceitual
ao faze-los. Além disso, gostaria de aprender a programar de forma mais profissional. Um colega meu
da física me apontou a possibilidade de ingresso no IME/BCC como aluno graduado. Fiz a prova no
final de 2008 e consegui entrar em uma das duas vagas oferecidas. Fiquei muito contente em voltar a
estudar. O problema estava em conciliar a faculdade com o trabalho, mas aos poucos tenho dado um
jeito.

\subsection{IME/BCC}

O ambiente acadêmico sempre me agradou muito, de forma que achei bastante tranquilo voltar a
estudar, principalmente no IME, que já era um ``velho conhecido''. Meu objetivo durante o BCC sempre
foi o de aproveitar o máximo possível assuntos que estavam ligados às minhas práticas no trabalho,
em especial, desenvolvimento de software de boa qualidade e aplicação de métodos de otimização. O
fato de eu ter tido uma boa formação prévia em matemática também me inclinou na direção da área de 
otimização, culminando na escolha do tema da monografia.

A formação que venho obtendo no IME tem sido bastante valiosa. De forma alguma teria conseguido
aprender as coisas que eu aprendi aqui se continuasse apenas estudando sozinho. Os professores
apresentam um conhecimento admirável em suas respectivas áreas de pesquisa. Isso tem mostrado para
mim que computação é {\bf muito} mais do que programação. Aliás, não optei por estudar computação
aos 18 anos por acreditar que ``apenas'' aprenderia a programar, uma ideia que não me agradava muito
naquela época.

Sem dúvida evolui muito ao longo desses quatro anos do ponto de vista de técnico: princípios de
desenvolvimento de bons algoritmos, estrutura de dados, programação orientada a objetos, arquitetura
e organização, sistemas operacionais, tudo isso era bastante novo e desconhecido para mim. O estudo
dessas disciplinas me possibilitou obter a base conceitual que procurava.

\subsection{Disciplinas}

Do ponto de vista de formação básica acho essencial o conhecimento de cálculo, álgebra linear e
estatística. Acredito que essas matérias já mais podem ser deixadas de fora do currículo da Ciência
da Computação, sendo fundamentais a qualquer um que deseja algo mais do que simplesmente programar.
Gostei de ter cursado as disciplinas de Álgebra I e II (não significa que eu tenha gostado de fazer
as provas e as listas de exercícios).

Do ponto de vista conceitual da computação em si, acredito que as disciplinas mais importantes
foram:

\begin{itemize}
	\item Princípios de Desenvolvimento de Algoritmos: por trazer ideias sobre recursividade, 
	{\it design} eficiente e análise.
	\item Estrutura de Dados: por ensinar o uso de estruturas eficientes e mergulhar a fundo na
	utilização da linguagem C, o pai de todas as linguagens.
	\item Análise de Algoritmos: ``A análise de algoritmos é uma disciplina de engenharia. Um
	engenheiro civil, por exemplo, tem métodos e tecnologia para prever o comportamento de uma 
	estrutura antes de construí-la. Da mesma forma, um projetista de algoritmos deve ser capaz de 
	prever o comportamento de um algoritmo antes de implementá-lo.'' Acho que não preciso dizer mais
	nada.
	\item Sistemas Operacionais: por ``desmistificar'' para mim a ligação entre o software e a 
	máquina.
\end{itemize}

Do ponto de vista mais prático (programação e tecnologia), as disciplinas mais importantes para mim
foram:

\begin{itemize}
	\item Laboratório de Programação I e II: pela introdução de linguagens, ferramentas e tecnologias.
	Pelas experiências das implementações nos trabalhos.
	\item Engenharia de Software: por possibilitar a experiência de se trabalhar em equipe,
	projetar e implementar (bem) um sistema que atenda as necessidades do cliente. Faltou apenas 
	aprender como cobrar (\$) desse cliente...
\end{itemize}

O que realmente me despertou interesse durante esses quatro anos foram as disciplinas da área de
computação científica e otimização. Acho que é uma combinação perfeita entre teoria e prática. A
teoria sem dúvida é bastante elegante e a prática tem resultados de impacto na indústria. Assim,
as melhores disciplinas que eu cursei até agora, em ordem de preferência, são:

\begin{itemize}
	\item Programação Linear: a teoria algébrica é muito bonita, com interpretação geométrica e forte
	utilização prática.
	\item Métodos Numéricos em Álgebra Linear: por apresentar as bases dos algoritmos que manipulam
	matrizes, como é o caso da programação linear e dos métodos de elementos finitos, importantíssimos
	na engenharia.
	\item Algoritmos de Aproximação: abordagem teórica que exige uma análise refinada dos algoritmos
	propostos para resolver problemas difíceis, como os de escalonamento. 
	Vale mencionar a excelência das aulas da professora Cris.
\end{itemize}

\subsection{Projeto do TCC}

A inclinação pela área de otimização e a curiosa necessidade de otimização dos tripulantes na
companhia aérea onde eu trabalho me levaram a estudar o tema apresentado nesta monografia. Durante
os anos desenvolvi um relacionamento de amizade com os colegas Lucas e Renato. Sempre trabalhávamos
em equipe, durante os EPs e os trabalhos. Assim, naturalmente sugeri o tema para eles.

O nosso orientador, professor Alfredo, mostrou-se interessado em nos orientar após uma conversa com
o Lucas. Ele achou o tema interessante e pediu para que nos aprofundássemos mais no assunto. Fomos
pesquisando e estudando, e aos poucos todos ficaram bem interessados. Daí resultou a nossa proposta
de trabalho original. O professor Alfredo sempre se mostrou bastante entusiasmado e nos guiou na
condução das decisões sobre o projeto e sobre as implementações a serem realizadas, mostrando ser
uma pessoa bastante acessível.

Ficamos felizes ao ver os resultados obtidos e tenho certeza de que fizemos um bom trabalho. Ainda
há o que melhorar e o professor Alfredo nos sugeriu até escrever um artigo para publicar nossos
resultados, principalmente se conseguirmos realizar o que está previsto nas perspectivas futuras.

\subsection{Comentários Finais}

O curso do BCC é puxado. Achei mais difícil do que a física; talvez seja a minha idade mais
avançada. São muitos trabalhos. São muitas provas! O conteúdo das disciplinas é difícil e as vezes
fazer a cobrança em prova seja complicado, ainda mais quando são três delas em um mesmo semestre.
Não sei se essa abordagem é válida. De qualquer forma, como já diria meu orientador no IFUSP: ``Todo
aprendizado deve ser traumático!''

Percebi que Ciência da Computação envolve muitas áreas e muitas pessoas com interesses bem
distintos. Muitos querem programar jogos, outros querem provar teoremas, outros querem desenvolver
sistemas web, outros querem ficar ricos, etc. Será que não seria interessante criar habilitações
específicas dentro do BCC para que os alunos, a partir do segundo ano, pudessem se aprofundar mais
em suas áreas de interesse? Gostaria de deixar aqui essa sugestão.














