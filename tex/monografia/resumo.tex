\chapter*{Resumo}
\thispagestyle{empty}

Uma viagem (ou {\it pairing}) é definida como uma sequência encadeada de voos, satisfazendo uma
série de restrições impostas pela legislação, que tem início e término na base contratual da
tripulação. No problema da determinação de viagens (PDV) tem-se como entrada uma lista de voos
oferecidos e deseja-se obter uma partição dessa lista em um conjunto de viagens viáveis. A partição
deve ser feita de forma a minimizar o custo, impondo que cada voo seja coberto exatamente uma vez
por alguma viagem da solução. O PDV pode ser formulado como um problema de otimização combinatória
NP-difícil conhecido por {\it set partition}. Nesta monografia analisamos o PDV no contexto
brasileiro, \ie, empregando-se as regras e a estrutura de custo adotada no Brasil para a geração de
viagens. Resolvemos o problema para algumas instâncias de teste fornecidas por companhias aéreas
brasileiras. Implementamos e aplicamos três heurísticas estudadas na literatura: um procedimento de
busca local, um algoritmo genético híbrido e um método de geração de colunas. As análises dos 
resultados obtidos indicam uma boa qualidade das soluções encontradas.


