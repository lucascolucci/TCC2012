\newpage
\section{Lucas Rodrigues Colucci}
\label{sec:lucas_subjetiva}

Até meu irmão ingressar no curso de Bacharelado em Ciências da Computação no IME, os programas de
computador funcionavam, na minha visão, através de mágica. Após seu ingresso, percebi que não era
magia e sim um monte de letras e números que, até então, não faziam sentido. Decidi que não
cometeria o mesmo ``erro'' e, caso não passasse na sonhada Engenharia Mecânica da POLI, faria um ano
de cursinho.

Não passei na POLI e fiquei em uma posição na qual não teria chances de ingresso em nenhuma das
quatro chamadas. Comecei o cursinho e, logo no primeiro dia, fiquei entediado com a rotina de
estudos. No terceiro dia um amigo me informou que eu havia passado na quarta chamada do BCC, e,
devido à insatisfação com o cursinho, comecei a pensar no assunto.

Pesquisei sobre a carreira de cientistas da computação e notei a possibilidade de uma boa
remuneração, e, em algumas empresas, de um estilo de trabalho descontraído. Optei por enfrentar o
desafio e ingressar no curso. Imaginava que, caso não gostasse do curso, poderia pedir transferência
para a POLI.

\subsection{IME/BCC}

Ao longo desses quatro anos, pensei diversas vezes em desistir do curso devido à sua dificuldade. No
entanto, após cursar as disciplinas de física, meu sonho de cursar engenharia já não existia mais e
decidi concluir a faculdade.

Com o tempo, fui pegando gosto pela programação e, aliado ao envolvimento com a atlética e com o
time de basquete, a vida no IME tornou-se mais fácil e agradável.


\subsection{Disciplinas}

Ao longo do curso, me deparei com disciplinas em que não entendia o por que de serem ministradas
para o BCC. No entanto, hoje vejo que todas foram de suma importância para minha formação, seja na
construção do raciocínio lógico ou na teoria da programação.

A seguir cito as matérias que considero mais importantes para a minha formação:

\begin{itemize} 
	\item Introdução à Computação: ensina e desmistifica o que é a programação;
	\item Princípios de Desenvolvimento de Algoritmos: promove a aprendizagem de alguns dos tópicos
mais importantes da computação como ordenação e lista ligada; 
	\item Análise de Algoritmos: apesar de ser uma matéria mais matemática, após a disciplina damos mais ênfase à eficiência do software durante seu desenvolvimento;
	\item Algoritmos em Grafos: essencial para resolver muitos problemas reais, inclusive o problema        de geração de viagens de nosso trabalho;
	\item Programação Linear: matéria mais interessante do curso. Ensina a arte da otimização, assunto de grande interesse para a maioria das empresas;
	\item Laboratório de Programação I e II e Engenharia de Software: primeiras oportunidades de trabalhar em programas extensos e em grupo; 
	\item Laboratório de Programação Extrema: disciplina na qual simulamos o trabalho em uma empresa
de software, com clientes e cobranças. Essencial para aprendermos como agir em diversas situações e
aperfeiçoar relações interpessoais;
\end{itemize}

\subsection{Projeto do TCC}

Logo no começo do curso, devido à necessidade de tirar dúvidas com outros alunos, comecei a me
relacionar com dois alunos do BCC: Daniel e Renato. A partir daí estudávamos e fazíamos trabalhos
sempre juntos.

No começo deste ano, após nos matricularmos na disciplina do TCC, começamos a pensar em um projeto
no qual pudéssemos trabalhar juntos. Após algumas idéias descartadas, o Daniel nos apresentou a
oportunidade de fazer um projeto muito interessante relacionado ao seu trabalho. Fomos
comentar a idéia com nosso orientador, o professor Alfredo, e ele aprovou a idéia.

Fizemos muitas pesquisas, lemos muitos papers sobre o assunto e começamos a implementação. Após
muito trabalho, terminamos o TCC e tivemos ótimos resultados. Isso fez todos os quatro anos de
graduação valerem a pena, mostrando que estamos preparados para enfrentar os mais difíceis desafios.

\subsection{Comentários Finais}

O que mais senti falta durante a minha formação foi do incentivo, por parte dos professores, de
realizarmos trabalhos em grupo. O mundo profissional exige uma boa relação entre pessoas, e isso não
é praticado no IME. Apesar disso, percebo que muitas das dificuldades geradas pelo curso foram
essenciais para meu amadurecimento e capacidade de enfrentar desafios.

Gostaria de deixar aqui meus agradecimentos aos coautores desse projeto, Daniel e Renato, os quais
me ajudaram ao longo de toda a graduação e se tornaram grandes amigos. Agradeço também aos meus
pais, Miguel e Angela, que sempre me incentivaram a continuar no curso; ao meu irmão, Thiago, que,
além de me incentivar, me ajudou em muitas das matérias; ao nosso orientador, Alfredo, que se
mostrou um grande professor e nos ajudou em todos os problemas do projeto; e à minha namorada e
melhor amiga, Aline, que mostrou-me que todos temos dificuldades e que o segredo da felicidade está
em como nos comportamos perante a elas.