\zerar
\chapter{Conclusão}
\label{cap:conclusoes}

Com relação a análise preliminar apresentada na Seção~\ref{sec:analise_p}, concluímos que o
procedimento de geração de viagens leva a um número gigantesco de variáveis, mesmo para um pequeno
número de pernas (Figura~\ref{fig:pairings}). Isso porque a natureza combinatória do problema leva o
algoritmo de busca a explorar diversas possibilidades, principalmente em uma rede como a da ponte
aérea, onde existem diversas possibilidades de conexão toda vez que se chega em uma das localidades.
Além disso, essas possibilidades se multiplicam quando consideramos um maior número de jornadas
permitidas (\verb|MAX_DUTIES|). Apesar disso, a geração de viagens ainda se fez em tempo aceitável,
podendo ser aplicada para redes maiores (Figura~\ref{fig:generation}).

Entretanto, quando esse número enorme de variáveis é levado ao otimizador, o tempo de processamento
se torna impraticável. Para se certificar disso, basta extrapolar as curvas obtidas nas
Figuras~\ref{fig:glpk} e~\ref{fig:cplex}. {\bf Uma tentativa de resolução de uma instância da
ponte-aérea contendo 40 etapas, não pode ser resolvida mesmo após 12 horas de processamento.} Ainda
assim, ficamos surpresos com a capacidade do otimizador resolver instâncias com um número de
variáveis da ordem de $10^6$ em tempo aceitável (resultados da Tabela~\ref{tab:resultados}).

A análise preliminar então nos mostra que o método de ``gerar-e-otimizar'' para resolver o modelo
(\ref{eq:sppv}) não é adequado para resolver o problema de forma geral. Em particular, das milhares
de variáveis geradas, apenas poucas delas são escolhidas para entrar na solução final, como se pode
observar da Tabela~\ref{tab:resultados}. Isso indica que o procedimento de geração explícita de 
variáveis não é adequado, pois muitas delas não servem para nada. Um procedimento mais inteligente
seria o de gerar apenas variáveis ``boas'', ou seja, com grande chance de aparecerem na solução 
final. O método de geração de colunas desenvolve essa ideia e será explorado futuramente.

Com relação aos resultados obtidos utilizando o modelo {\it set cover}, observamos que como as 
colunas associadas às variáveis $y_i$ de {\it deadheading} (reveja o formulação (\ref{eq:scpdv})) 
foram ajustadas com preços altos, e como os problemas analisados eram viáveis do ponto de vista do 
{\it set partition}, o otimizador conseguiu encontrar as mesmas soluções obitidas sem a presença de 
{\it deadheading}. Concluímos que a utilização de {\it deadheading} só será realizada se for 
estritamente necessária para viabilidade do problema.

%%%%%%%%%%%%%%%%%%%%%%%%%%%%%%%%%%%%%%%%%%%%%%%%%%%%%%%%%%%%%%%%%%%%%%%%%%%%%%%%%%%%%%%%%%%%%%%%%%%%

\section{Comparação Entre Heurísticas}
\label{sec:comparacao}

%%%%%%%%%%%%%%%%%%%%%%%%%%%%%%%%%%%%%%%%%%%%%%%%%%%%%%%%%%%%%%%%%%%%%%%%%%%%%%%%%%%%%%%%%%%%%%%%%%%%

\section{Perspectivas Futuras}
\label{sec:perspectivas}


