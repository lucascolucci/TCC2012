\zerar
\chapter{Conclusão}
\label{cap:conclusao}

Com relação a análise preliminar apresentada na Seção~\ref{sec:preliminar}, concluímos que o
procedimento de geração de viagens leva a um número gigantesco de variáveis, mesmo para um pequeno
número de pernas (Figura~\ref{fig:pairings}). Isso porque a natureza combinatória do problema leva o
algoritmo de busca a explorar diversas possibilidades, principalmente em uma rede como a da ponte
aérea, onde existem diversas possibilidades de conexão toda vez que se chega em uma das localidades.
Além disso, essas possibilidades se multiplicam quando consideramos um maior número de jornadas
permitidas (\verb|MAX_DUTIES|). Apesar disso, a geração de viagens ainda se fez em tempo aceitável,
podendo ser aplicada para redes maiores (Figura~\ref{fig:generation}).

Entretanto, quando esse número enorme de variáveis é levado ao otimizador, o tempo de processamento
se torna impraticável. Para se certificar disso, basta extrapolar as curvas obtidas nas
Figuras~\ref{fig:glpk} e~\ref{fig:cplex}. Uma tentativa de resolução de uma instância da ponte-aérea
contendo 40 etapas, não pode ser resolvida mesmo após 12 horas de processamento. Ainda assim,
ficamos surpresos com a capacidade do otimizador resolver instâncias com um número de variáveis da
ordem de $10^6$ em tempo aceitável (resultados da Tabela~\ref{tab:resultados}).

A análise preliminar então nos mostra que o método de ``gerar-e-otimizar'' não é adequado para
resolver o problema de forma geral. Em particular, das milhares de variáveis geradas, apenas poucas
delas são escolhidas para entrar na solução final, como se pode observar da
Tabela~\ref{tab:resultados}. Isso indica que o procedimento de geração explícita de variáveis não é
adequado, pois muitas delas não servem para nada. Um procedimento mais inteligente seria o de gerar
apenas variáveis ``boas'', ou seja, com grande chance de aparecerem na solução final. O método de
geração de colunas é o que desenvolve essa ideia.

Com relação aos resultados exatos obtidos utilizando o modelo {\it set cover} (\ref{eq:scpdv}), 
observamos que como as colunas associadas às variáveis $y_i$ foram ajustadas com preços altos, e 
como os problemas analisados eram viáveis do ponto de vista do {\it set partition}, o otimizador 
encontrou as mesmas soluções que seriam obitidas sem a presença de {\it deadheading}. Assim, a 
presença de {\it deadheading} na solução só existirá se for estritamente necessária para 
viabilidade do problema. Infelizmente apenas os problemas P1, P2 e P3 da Tabela~\ref{tab:problemas}
puderam ser resolvidos exatamente.

Analisando a heurística da busca local, podemos tirar mais algumas conclusões com relação ao
parâmetro $k$ de sua implementação. Em nossos testes, utilizamos três valores distintos para $k$ 
e com base nos dados da Tabela~\ref{tab:comparacao} e gráficos da Figura~\ref{fig:ls_results} 
podemos afirmar:

\begin{itemize}
\item $k = 2$: Pode ser visto facilmente nos gráficos que o algoritmo converge para mínimos locais
rapidamente, apresentando ganhos pequenos por iteração. Além disso, muitas iterações não resultam em
melhoria pois os subproblemas resultantes da escolha de duas viagens são normalmente inviáveis ou já
são ótimas; 
\item $k = 3$: Mostrou-se o valor mais eficiente para k pois apresentou uma boa convergência e
iterações mais rápidas do que $k = 4$. Para obter soluções tão boas quanto $k = 4$, torna-se
necessário um número maior de iterações.
\item $k = 4$: Converge em poucas iterações em direção ao ótimo mas cada iteração necessita de um
tempo maior de processamento.
\end{itemize}

O algoritmo genético híbrido proposto apresenta dependência sensível de desempenho com relação a
seus parâmetros. Em particular, o número de vezes, $L$, que cada indivíduo da população inicial é
iterado pelo método de busca local, foi variado em nossos testes. Para cada valor de $L$ utilizado,
chegamos as seguintes conclusões:

\begin{itemize}
\item $L = 1$: Uma iteração por indivíduo não foi suficiente para gerar melhorias significativas na
convergência;
\item $L = 5$: Tornou mais rápida a convergência, mostrando a utilização de busca local tem um
impacto positivo nos resultados;
\item $L = 10$: Para instâncias pequenas teve influência similar ao $L = 5$, no entanto para a
instância 73G\_340, seu desempenho foi muito superior. Isso indica que quanto maior o número de
voos, maior é o valor de $L$ que maximiza os ganhos de performance.
\end{itemize}

O método de geração de coluna implementado se mostrou bastante eficiente para a resolução da
relaxação linear associada ao problema. Entretanto, para as instâncias grandes, os subconjuntos de
viagens geradas não foram pequenos o suficiente para que pudéssemos resolver o problema linear
inteiro com essas variáveis. Ou seja, não foi possível obter uma solução inteira utilizando as
colunas geradas.

Heurísticas ainda podem ser aplicadas no sentido de se diminuir o subconjunto de viagens geradas,
possibilitando a resolução do problema inteiro. Por exemplo, podemos restringir o número máximo de
caminhos encontrados até cada nó do grafo durante a resolução do {\it pricing problem}. Podemos
ainda considerar no final apenas as colunas com custo reduzido abaixo de um valor limite ({\it
cutoff}). Não fizemos a implementação dessas heurísticas nesse trabalho.

Para resumir, listamos a seguir em forma de tópicos algumas conclusões gerais com relação ao
problema estudado neste projeto:

\begin{itemize}
\item O problema de geração de viagens é muito mais complexo do que aparenta. 
\item O problema realmente necessita de métodos aproximados devido à incapacidade dos métodos exatos
em resolver instancias grandes.
\item Uma solução otimizada realmente pode gerar uma grande economia à empresa aérea.
\item Ainda há muito espaço para pesquisas nessa área. 
\item O problema de otimização de viagens é apenas parte de um problema maior que seria o
escalonamento dos tripulantes.
\end{itemize}

%%%%%%%%%%%%%%%%%%%%%%%%%%%%%%%%%%%%%%%%%%%%%%%%%%%%%%%%%%%%%%%%%%%%%%%%%%%%%%%%%%%%%%%%%%%%%%%%%%%%

\section{Perspectivas Futuras}
\label{sec:perspectivas}

\begin{itemize} 
\item Implementação de um esquema branch-and-price para obtenção de solução inteira a partir da
geração de colunas.
\item Combinação e paralelização das heurísticas estudadas, explorando os pontos fortes de cada uma
delas.
\item Possível sistema comercial.
\end{itemize}

%%%%%%%%%%%%%%%%%%%%%%%%%%%%%%%%%%%%%%%%%%%%%%%%%%%%%%%%%%%%%%%%%%%%%%%%%%%%%%%%%%%%%%%%%%%%%%%%%%%%